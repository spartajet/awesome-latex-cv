% Awesome CV LaTeX Template
%
% This template has been downloaded from:
% https://github.com/huajh/huajh-awesome-latex-cv
%
% Author:
% Junhao Hua


%Section: Work Experience at the top
\sectionTitle{项目经历}{\faCode}
 
\begin{experiences}
			
 \experience
    {2016年10月}   {下一代齿轮测量的基础理论与关键技术研究(国基金重点项目)}{北京工业大学}{主要项目成员}
    {至今} {
                      \begin{itemize}
                        \item 基于全信息的齿轮精度评价体系与齿轮广义误差理论研究
                        \item 齿轮工艺误差分析、溯源及机床反调的解耦新方法研究
                        \item 齿轮动态特性预报研究
                        \item 齿轮测量数据标准化接口及齿轮测量云平台研究
                        \item 下一代齿轮测量实验验证系统构建
                        \item 作为项目核心成员,负责齿轮全信息精度评价体系的数据获取及数据格式标准和云平台建设。基于英国RPI 高精度气浮转台和 Keyence 线结构光传感器搭建了齿轮点云测量系统,开发了齿轮点云数据测量和数据分析系统,该系统目前是我国第一套齿轮点云测量系统,国际上比较先进的齿轮测量装置,能够获取齿轮全信息数据,为新的齿轮评定方法和理论提供了基础。在此基础上,基于spring-cloud开发了齿轮测量云平台,实现了计算资源的云端调度功能,该云平台是工业4.0在齿轮测量领域的典型应用。在云平台基础上,开发并制定了基于xml的齿轮数据格式标准,是目前齿轮测量领域的第一个数据标准,填补了国内空白,该标准即将
                        成为国家标准     
                      \end{itemize}
                    }
                    {精密仪器, 测控系统集成, 点云测量技术, C\#, Python, Java}
  \emptySeparator
  \experience
    {2015年6月} {汽车齿轮快速检测及高效配对系统的研发(国家科技重大专项)}{北京工业大学}{主要成员}
    {2017年6月}    {
                      \begin{itemize}
                        \item 实现汽车齿轮快速检测的新原理、齿轮全误差获取方法、误差评定方法、系统标定方法研究
                        \item 实现自动上下料、快速测量、分选三大功能的工作原理及其实现研究
                        \item 以及采用监督学习(KNN,SVM)对每帧图像的字典分类                    
                        \item 批量齿轮配对机理、智能配对技术、配对质量评价方法研究
                        \item 作为项目核心成员,批量齿轮配对机理、智能配对技术、配对质量评价方法研究。在成都,北京和哈尔滨三地陆续做齿轮配对实验上千次实验,开发了基于tensorflow和pytorch的齿轮智能配对技术,率先将机器学习应用到齿轮测量过程中                      
                      \end{itemize}
                    }
                    {齿轮测量, 齿轮配对, 齿轮评价, tensorflow, Pytorch}
	
  \emptySeparator
  \experience
  {2016年6月} {齿轮形性综合试验台(国家自然科学基金)}{北京工业大学}{主要成员}
  {2017年11月 }    {
				  	\begin{itemize}
				  		\item 齿轮传动误差测量
				  		\item 齿轮振动噪声测试
				  		\item 探索齿轮振动噪声和传动误差之间的关系
				  		\item 作为核心成员,开发了基于NI PCI噪声和振动测量板卡的齿轮振动噪声测量系统,研究了振动噪声与传动误差的关系
				  	\end{itemize}
				  }
				  {齿轮形性, 传动误差, 振动噪声, NI, 测试板卡}

  \emptySeparator
  \experience
  {2017年6月} {高精度气浮转台云计算平台研发}{北京工业大学}{独立开发}
  {2017年12月}    {
  	\begin{itemize}
  		\item 气浮转台云端系统搭建
  		\item 气浮转台控制下位机开发
  		\item 独立完成该项目,创造性的在气浮转台原有闭环系统上加入了云端控制组件,为齿轮点云测量提供了平台基础。采用了Spring Cloud 构建了实验室的仪器设备云计算平台。
  	\end{itemize}
  }
  {高精度气浮转台, 测量云计算,SpringCloud}
		  				  
  \emptySeparator
  \experience
  {2014年12月} {齿轮传动测量系统软件开发}{北京工业大学}{独立研发}
  {2015年3月}    {
				  	\begin{itemize}
				  		\item 面齿轮传动误差测量软件  
				  		\item 面齿轮接触分析TCA的相关研究                
				  		\item 独立研发了面齿轮传动误差测量软件
				  	\end{itemize}
				  }
				  {面齿轮, 传动误差}
	\emptySeparator
  \experience
  {2015年8月} {齿轮缺陷调谐共振检测机理及关键技术(青年科学基金项目)}{北京工业大学}{参与研发}
  {2016年10月}    {
				  	\begin{itemize}
				  		\item 齿轮缺陷调谐共振机理检测机理研究                  
				  		\item 齿轮缺陷调谐共振快速检测系统研究
				  		\item 基于调谐共振的齿轮缺陷检测实验研究
				  		\item 参与研发了项目中齿轮振动部分的相关研发
				  	\end{itemize}
				  }
				  {齿轮, 齿轮共振, 齿轮缺陷}
  \emptySeparator
  \experience
  {2015年10月} {PCI板卡驱动开发}{北京工业大学}{独立研发}
  {2015年11月}    {
				  	\begin{itemize}
				  		\item 独立开发了某型自定义PCI板卡的驱动开发  
				  	\end{itemize}
				  }
          {PCI驱动开发}
  \emptySeparator
  \experience
  {2013年7月} {智能康复机器人研发}{北方工业大学}{本科创新项目}
  {2014年6月}    {
				  	\begin{itemize}
				  		\item 开发智能康复机器人的电气控制系统和软件控制系统
				  		\item 承担了部分机械设计任务
				  	\end{itemize}
				  }
				  {电气控制系统, 软件,测试板卡}
	\emptySeparator
  \experience
  {2009年7月} {航天发射靶场地勤监控系统研发}{中国人民解放军装甲兵工程学院}{本科毕业设计}
  {2012年5月}    {
				  	\begin{itemize}
				  		\item 独立完成某航天发射靶场子系统监控系统,已应用在实际发射过程中
				  		\item 作为本科毕业设计,系统的训练了科研方法和素质
				  	\end{itemize}
				  }
          {航天, 环境监控}
  \emptySeparator
  \experience
  {2018年7月} {国家发改委营商环境评价核心算法开发}{北京}{}
  {2019年5月}    {
				  	\begin{itemize}
				  		\item 独立完成国家发改委营商环境评价核心算法开发
				  		\item 承担了部分云平台开发工作
				  	\end{itemize}
				  }
				  {Java, SpringCloud}
		
\end{experiences}
